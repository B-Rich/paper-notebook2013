\documentclass[12pt]{article}

\usepackage[english]{babel}
\usepackage[utf8x]{inputenc}
\usepackage{amsmath}
\usepackage{graphicx}
\usepackage[margin=1.0in]{geometry}
\usepackage{authblk}

\title{The IPython Notebook: open, reproducible scientific computing}
\author[1]{Matthias Bussonnier}
\author[2]{Jonathan Frederic}
\author[3]{Bradley M. Froehle}
\author[2]{Brian E. Granger}
\author[3]{Paul Ivanov}
\author[3]{Thomas Kluyver}
\author[3]{Fernando Perez}
\author[3]{Benjamin Ragan-Kelley}
\author[2]{Zachary Sailer}
\affil[1]{Affiliation of Matthias}
\affil[2]{Cal Poly State University}
\affil[3]{University of CA, Berkeley}

\renewcommand\Authands{ and }

\begin{document}
\maketitle

\begin{abstract}
While computing has become a foundation of all it is challenging for researchers . 
\end{abstract}

\section{Introduction}

\section{The lifecycle of research}

\section{The IPython Notebook}

\subsection{Web application}

\subsection{Notebook document format}

\subsection{Installation}

\section{Collaboration}

\section{Broader ecosystem}

\section{Future directions}

\end{document}